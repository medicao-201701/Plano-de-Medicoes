\chapter{Métricas}
	
	Para chegar aos objetivos levantados, a equipe teve que levantar métricas de forma que identifique os pontos fracos e fortes do projeto, a fim de que uma boa entrega seja garantida.
	Nesse capítulo serão apresentadas as métricas responsáveis por responder as questões levantadas em relação aos objetivos a serem alcançadas.

\section{M.1.1.2 - Code Smells} % Stylesheet

	\begin{tabular}{ |p{5cm}|p{5cm}|  }
	 \hline
	 Objetivo da Medição 		& 	   \\
	 \hline
	 Fórmula		& 		\\
	 \hline
	 Escala da Medição 		& 		 \\
	 \hline
	 Coleta		& 		\\
	 \hline
	 Análise		& 		 \\
	 \hline
	 Meta		& 		 \\
	 \hline
	\end{tabular}

\section{M.1.2.1 - Complexidade NPath}

	\begin{tabular}{ |p{5cm}|p{5cm}|  }
	 \hline
	 Objetivo da Medição 		& 	  Garantir que o código tenha a menor complexidade possível em caso de necessidade de manutenção. \\
	 \hline
	 Fórmula		& 		\[ c = \prod{n} \] \begin{itemize} \item c = Complexidade NPath 
	 \item n = Número de estruturas de decisão \end{itemize}\\
	 \hline
	 Escala da Medição 		& 		Absoluta \\
	 \hline
	 Coleta		& 		\begin{itemize} \item Responsável: Equipe de medição \item: Periodicidade: A cada marco \item Procedimentos: Análise com as ferramentas cLint. A coleta da complexidade NPath analisa a profundidade das estruturas de decisão. \end{itemize} \\
	 \hline
	 Análise		& 		\begin{itemize} \item Responsável: Equipe de medição \item Procedimentos: Quanto maior a profundidade e aninhamento entre essas estruturas, mais difícil a compreensão de código, o que dificulta na manutenção e aumenta o custo da mesma. \end{itemize} \\
	 \hline
	 Meta		& 	1-200, complexidade aceitável. [Referência]	 \\
	 \hline
	\end{tabular}


\section{M.1.3.1 - Duplicação de código} 

	\begin{tabular}{ |p{5cm}|p{5cm}|  }
	 \hline
	 Objetivo da Medição 		& 	   \\
	 \hline
	 Fórmula		& 		\\
	 \hline
	 Escala da Medição 		& 		 \\
	 \hline
	 Coleta		& 		\\
	 \hline
	 Análise		& 		 \\
	 \hline
	 Meta		& 		 \\
	 \hline
	\end{tabular}

\section{M.2.1.1 - Horas trabalhadas} % Dedicação

	\begin{tabular}{ |p{5cm}|p{5cm}|  }
	 \hline
	 Objetivo da Medição 		& https://dev9.com/blog-posts/2015/1/the-myth-of-developer-productivity 	   \\
	 \hline
	 Fórmula		& 		\\
	 \hline
	 Escala da Medição 		& 		 \\
	 \hline
	 Coleta		& 		\\
	 \hline
	 Análise		& 		 \\
	 \hline
	 Meta		& 		 \\
	 \hline
	\end{tabular}

	\section{M.2.1.1 - Evolucao de conhecimento} % Dedicação

	\begin{tabular}{ |p{5cm}|p{5cm}|  }
	 \hline
	 Objetivo da Medição 		& Verificar conhecimento adquirido entre os marcos	   \\
	 \hline
	 Fórmula		& 		\\
	 \hline
	 Escala da Medição 		& 		 \\
	 \hline
	 Coleta		& 		\\
	 \hline
	 Análise		& 		 \\
	 \hline
	 Meta		& 		 \\
	 \hline
	\end{tabular}

	\section{M.2.1.1 - Active days} % Dedicação

	\begin{tabular}{ |p{5cm}|p{5cm}|  }
	 \hline
	 Objetivo da Medição 		& Verificar conhecimento adquirido entre os marcos	   \\
	 \hline
	 Fórmula		& 		https://blog.gitprime.com/5-developer-metrics-every-software-manager-should-care-about\\
	 \hline
	 Escala da Medição 		& 		 \\
	 \hline
	 Coleta		& 		\\
	 \hline
	 Análise		& 		 \\
	 \hline
	 Meta		& 		 \\
	 \hline
	\end{tabular}

\section{M.3.1.1 - Eficiência da equipe - Accuracy of Estimation} % Eficiencia para produzir artefatos

	\begin{tabular}{ |p{5cm}|p{5cm}|  }
	 \hline
	 Objetivo da Medição 		& https://dev9.com/blog-posts/2015/1/the-myth-of-developer-productivity  	   \\
	 \hline
	 Fórmula		& 		\\
	 \hline
	 Escala da Medição 		& 		 \\
	 \hline
	 Coleta		& 		\\
	 \hline
	 Análise		& 		 \\
	 \hline
	 Meta		& 		 \\
	 \hline
	\end{tabular}


\section{M.3.2.1 - Estimativa de entrega} % A estimativa de entrega de artefato corresponde com a atual taxa de entrega dos artefatos

	\begin{tabular}{ |p{5cm}|p{5cm}|  }
	 \hline
	 Objetivo da Medição 		& Linhas de código (LOC)	   \\
	 \hline
	 Fórmula		& 		\\
	 \hline
	 Escala da Medição 		& 		 \\
	 \hline
	 Coleta		& 		\\
	 \hline
	 Análise		& Número de linhas de código entregue, desconsiderando linhas em branco e comentários \\
	 \hline
	 Meta		& 		 \\
	 \hline
	\end{tabular}