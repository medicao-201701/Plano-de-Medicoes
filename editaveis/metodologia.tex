\chapter{Métricas}
	
	Para chegar aos objetivos levantados, a equipe teve que levantar métricas de forma que identifique os pontos fracos e fortes do projeto, a fim de que uma boa entrega seja garantida.
	Nesse capítulo serão apresentadas as métricas responsáveis por responder as questões levantadas em relação aos objetivos a serem alcançadas.

\section{M.1.1.2 - Code Smells} % Stylesheet

	\begin{tabular}{ |p{4cm}|p{8cm}|  }
	 \hline
	 Objetivo da Medição 		& 	 Assegurar legibilidade do código  \\
	 \hline
	 Fórmula		& 	Não se aplica	\\
	 \hline
	 Escala da Medição 		& 	Absoluta	 \\
	 \hline
	 Coleta		& 	\begin{itemize} \item Responsável: Equipe de medição \item Periodicidade: a cada marco da disciplina \item Procedimento: Será utilizado da ferramenta cLint para realizar a análise estática nos métodos, para detectar métodos e lista de parâmetros longos. \end{itemize} \\
	 \hline
	 Análise		& 	\begin{itemize} \item Responsável: Equipe de medição \item Procedimentos: Quanto maior o número de bad smells, pior a legibilidade do código \end{itemize} 	 \\
	 \hline
	 Meta		& 	BadSmells <= 10	 \\
	 \hline
	\end{tabular}

\section{M.1.2.1 - Complexidade NPath}

	\begin{tabular}{ |p{4cm}|p{8cm}|  }
	 \hline
	 Objetivo da Medição 		& 	  Garantir que o código tenha a menor complexidade possível em caso de necessidade de manutenção. \\
	 \hline
	 Fórmula		& 		\[ c = \prod{n} \] \begin{itemize} \item c = Complexidade NPath 
	 \item n = Número de estruturas de decisão \end{itemize}\\
	 \hline
	 Escala da Medição 		& 		Absoluta \\
	 \hline
	 Coleta		& 		\begin{itemize} \item Responsável: Equipe de medição \item: Periodicidade: A cada marco \item Procedimentos: Análise com as ferramentas cLint. A coleta da complexidade NPath analisa a profundidade das estruturas de decisão. \end{itemize} \\
	 \hline
	 Análise		& 		\begin{itemize} \item Responsável: Equipe de medição \item Procedimentos: Quanto maior a profundidade e aninhamento entre essas estruturas, mais difícil a compreensão de código, o que dificulta na manutenção e aumenta o custo da mesma. \end{itemize} \\
	 \hline
	 Meta		& 	1-200, complexidade aceitável. [Referência]	 \\
	 \hline
	\end{tabular}


\section{M.1.3.1 - Duplicação de código} 

	\begin{tabular}{ |p{4cm}|p{8cm}|  }
	 \hline
	 Objetivo da Medição 		& 	Assegurar a não repetição de código já criado em outros módulos, de forma que facilite a manutenção.   \\
	 \hline
	 Fórmula		& 	Quantidade de linhas de código duplicado / Quantidade total de linhas de código	\\
	 \hline
	 Escala da Medição 		& Absoluta		 \\
	 \hline
	 Coleta		& 	\begin{itemize} \item Responsável: Equipe de medição \item Periodicidade: a cada marco da disciplina \item Procedimento: Será utilizado da ferramenta Simian para realizar a análise estática nos arquivos, para detectar a duplicação de código.\end{itemize}	\\
	 \hline
	 Análise		& 	\begin{itemize} \item Responsável: Equipe de medição \item Procedimentos: Quanto mais duplicação, mais tempo será necessário para dar manutenção em funções. \end{itemize}	 \\
	 \hline
	 Meta		& 	Duplicidade <= 10\%	 \\
	 \hline
	\end{tabular}

\section{M.2.1.1 - Horas trabalhadas} % Dedicação

	\begin{tabular}{ |p{4cm}|p{8cm}|  }
	 \hline
	 Objetivo da Medição 		& Determinar a quantidade de tempo em que cada integrante se dedicou por marco.	   \\ % https://dev9.com/blog-posts/2015/1/the-myth-of-developer-productivity 
	 \hline
	 Fórmula		&  Somatório da quantidade de horas	\\
	 \hline
	 Escala da Medição 		& Absoluta		 \\
	 \hline
	 Coleta		& 	\begin{itemize} \item Responsável: Equipe de medição \item Periodicidade: a cada marco da disciplina \item Procedimento: Será utilizado da ferramenta WakaTime para realizar a análise de quantas horas o integrante trabalhou por semana, e a equipe realizará o somatório do intervalo entre os marcos.\end{itemize}	\\
	 \hline
	 Análise		& 	\begin{itemize} \item Responsável: Equipe de medição \item Procedimentos: Quanto mais horas trabalhadas, maior a dedicação. \end{itemize}	 \\
	 \hline
	 Meta		& 	6 horas por semana	 \\
	 \hline
	\end{tabular}

	\section{M.2.1.1 - Evolucao de conhecimento} % Dedicação

	\begin{tabular}{ |p{4cm}|p{8cm}|  }
	 \hline
	 Objetivo da Medição 		& Verificar conhecimento adquirido entre os marcos	   \\
	 \hline
	 Fórmula		& 	Não se aplica	\\
	 \hline
	 Escala da Medição 		& 	Ordinal	 \\
	 \hline
	 Coleta		& 	\begin{itemize} \item Responsável: Equipe de medição \item Periodicidade: a cada marco da disciplina \item Procedimento: Será utilizado um questionário para que os desenvolvedores opinem sobre o quanto estão familiarizados com as tecnologias e o processo de desenvolvimento. \end{itemize}		\\
	 \hline
	 Análise		& 	\begin{itemize} \item Responsável: Equipe de medição \item Procedimentos: De um valor inteiro de 1 a 5, quanto maior o número, mais o desenvolvedor se diz familiarizado. \end{itemize}	 \\
	 \hline
	 Meta		& 	Conhecimento dos desenvolvedores em cada tecnologia >= 4	 \\
	 \hline
	\end{tabular}

	%\section{M.2.1.1 - Active days} % Dedicação

	%\begin{tabular}{ |p{5cm}|p{5cm}|  }
	% \hline
	% Objetivo da Medição 		&    \\
	% \hline
	% Fórmula		& 	Número de commits por dia	\\ % https://%blog.gitprime.com/5-developer-metrics-every-software-manager-should-care-about
	% \hline
	% Escala da Medição 		& 		 \\
	% \hline
	% Coleta		& 		\\
	% \hline
	% Análise		& 		 \\
	% \hline
	% Meta		& 		 \\
	% \hline
	%\end{tabular}

\section{M.3.1.1 - Eficiência da equipe - Accuracy of Estimation} % Eficiencia para produzir artefatos

	\begin{tabular}{ |p{4cm}|p{8cm}|  }
	 \hline
	 Objetivo da Medição 		&  	 Avaliar a capacidade da equipe em entregar os artefatos estipulados pelo docente.  \\ % https://dev9.com/blog-posts/2015/1/the-myth-of-developer-productivity 
	 \hline
	 Fórmula		& 	Quantidade de tempo definido - Quantidade de tempo gasta	\\
	 \hline
	 Escala da Medição 		& 	Racional	 \\
	 \hline
	 Coleta		& 	\begin{itemize} \item Responsável: Equipe de medição \item Periodicidade: a cada marco da disciplina \item Procedimento: Após o início de cada marco, estipular o tempo de cada tarefa e aferir o tempo de conclusão das mesmas. \end{itemize}	\\
	 \hline
	 Análise		& 	\begin{itemize} \item Responsável: Equipe de medição \item Procedimentos: Se o valor for positivo, a eficiência está boa, caso seja negativa, o time precisa se esforçar mais. \end{itemize}	 \\
	 \hline
	 Meta		& 	Eficiência >= 0	 \\
	 \hline
	\end{tabular}


%\section{M.3.2.1 - Estimativa de entrega} % A estimativa de entrega de artefato corresponde com a atual taxa de entrega dos artefatos?

%	\begin{tabular}{ |p{5cm}|p{5cm}|  }
%	 \hline
%	 Objetivo da Medição 		& 	   \\
%	 \hline
%	 Fórmula		& 		\\
%	 \hline
%	 Escala da Medição 		& 		 \\
%	 \hline
%	 Coleta		& 		\\
%	 \hline
%	 Análise		&  \\
%	 \hline
%	 Meta		& 		 \\
%	 \hline
%	\end{tabular}