\chapter{Questões}

\subsection{O.1 - Manutenibilidade}

	\begin{tabular}{ |p{6cm}|p{6cm}|  }
	 \hline
	 Foco na qualidade 		& 		Fatores de variação \\
	 \hline
	 \begin{itemize} \item Q1.1 Existe um padrão para estilo de código? \item Q1.2 Qual a complexidade de código? \item Q1.3 Qual a taxa de duplicação de código? \end{itemize} & \begin{itemize} \item Estudantes ficarem impossibilitados \item Prazo curto para entregas 
	 \item Conhecimento limitado da tecnologia \end{itemize}\\
	 \hline
	 Hipótese de BASELINE 		& 		Impactos nas hipóteses de BASELINE \\
	 \hline
	 \begin{itemize} \item \item Complexidade NPath: 0  \item Duplicação do código: 0\% \end{itemize} & \begin{itemize} \item Código complexo \end{itemize} \\
	 \hline
	\end{tabular}

\subsection{O.2 - Dedicação}

	\begin{tabular}{ |p{6cm}|p{6cm}|  }
	 \hline
	 Foco na qualidade 		& 		Fatores de variação \\
	 \hline
	 \begin{itemize} \item Q2.1 Qual a dedicação da equipe com o desenvolvimento? \end{itemize} & \begin{itemize} \item Equipamentos defeituosos \item Greve da UnB
	 \end{itemize}\\ %easter egg -> Potter gay.
	 \hline
	 Hipótese de BASELINE 		& 		Impactos nas hipóteses de BASELINE \\
	 \hline
	 \begin{itemize} \item Baixa: Horário das aulas semanais  \end{itemize} 		& 		\begin{itemize} \item Presença nas aulas \item Presença nas reuniões 
	 \item Ambiente de desenvolvimento configurado e funcionando \end{itemize} \\
	 \hline
	\end{tabular}

\subsection{O.3 - Incremento de produto}

	\begin{tabular}{ |p{6cm}|p{6cm}|  }
	 \hline
	 Foco na qualidade 		& 		Fatores de variação \\
	 \hline
	 \begin{itemize} \item Q2.1 Qual a eficiência do time para produzir artefatos? \item Q2.2 A estimativa de entrega de artefato corresponde com a atual taxa de entrega dos artefatos?\end{itemize} & \begin{itemize} \item Greve da UnB \item Equipamentos defeituosos \item Atividades externas à matéria \end{itemize}\\
	 \hline
	 Hipótese de BASELINE 		& 		Impactos nas hipóteses de BASELINE \\
	 \hline
	 \begin{itemize} \item Eficiência: 0(inicial) \end{itemize} 		& 		Impactos nas hipóteses de BASELINE \\
	 \hline
	\end{tabular}